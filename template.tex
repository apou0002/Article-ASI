\documentclass[a4paper,12pt]{article} % ce document est un article sur une feuille A4, police taille 12

\usepackage[utf8]{inputenc} % encodé en utf-8
\usepackage[T1]{fontenc} % compatible avec les accents

\usepackage[round]{natbib} % gestion des citations
\usepackage[french]{babel} % rédigé en français
\usepackage[hyphens]{url} % formatte les liens en autorisant la césure au niveau des traits d'union
\usepackage[pdftex,urlcolor=black,colorlinks=true,linkcolor=black,citecolor=black]{hyperref} % liens cliquables mais non colorés
\usepackage[top=3cm,bottom=4cm]{geometry} % gère les marges
\usepackage{graphicx} % gestion des images
\usepackage{array} % gestion des tableaux
\usepackage{csquotes} % gestion des guillemets
\usepackage{fourier} % utilise une autre police que celle par défaut (Computer Modern)
\usepackage{setspace}% gestion des interlignes
\onehalfspacing
% insérez ici d'autres extensions avec la commande \usepackage[options]{nom de l'extension}

\title{QUEL LANGAGE DE PROGRAMMATION CHOISIR POUR CREER UN LOGICIEL ?} % le titre de l'article
\author{Abdel Josselin POUAMOUN} % vos prénom et nom
\date{} % pas de date

\begin{document} % début du corps du texte
\maketitle % affiche le titre

\section{INTRODUCTION} % section 1
Un ordinateur est une machine électronique programmable servant au traitement de l'information. Il peut être  assimilé à un système produisant des résultats à partir : d'informations fournies , et de méthode de résolution permettant de traiter ces informations.
Les informations constituent des données et les méthodes de résolution aident à construire des algorithmes qui représentent l'enchaînement des actions à réaliser qui sont nécessaires à la résolution d'un problème donné et dont la traduction dans un langage de programmation permet de réaliser un programme, une application ou un logiciel.
Lorsque nait l’idée de conception d’un projet logiciel, la première question qu’on devrait se poser, c’est de savoir avec quel langage de programmation  on doit le développer ? 
C’est bien beau de vouloir programmer ou d’apprendre un langage de programmation, mais il faut savoir lequel choisir, et pour cela un ensemble de  critère devrait être examiné au préalable.
Écrire des programmes nécessite l'utilisation d'un langage de programmation (ou plusieurs). Il y a toute une pléthore de langages de programmation disponibles. Ces langages ne sont pas, forcement équivalents bien qu’il existe parfois certaines similitudes entre eux. Chaque langage possède ses avantages et ses inconvénients.
Faisons ainsi un tour des questions à se poser (critères à examiner) pour choisir correctement son langage de programmation.
Afin de choisir un langage ou une implémentation, un programmeur doit tenir compte d'un certain nombre paramètres :
\begin{itemize}
\item[$\bullet$]L’utilisabilité du langage
Facilité d'apprentissage, et facilité d'utilisation pour un programmeur expérimenté.
\item[$\bullet$]Les performances du langage
Rapidité d'exécution des programmes, rapidité d'exécution du compilateur,  stabilité (absence de défaut)…
\item[$\bullet$]La portabilité du langage sur différentes plateformes
Sa capacité à pouvoir être adapté plus ou moins facilement en vue de fonctionner dans différents environnements d'exécution. Les différences peuvent porter sur l'environnement matériel (processeur) comme sur l'environnement logiciel (système d'exploitation).
\item[$\bullet$]L’extensibilité du langage
Perspectives d'évolution du langage ou de son implémentation, existence de bibliothèques de fonctions, de classes, etc.,
\item[$\bullet$]La pérennité du langage
Pérennité du fabricant, pérennité du langage, pérennité de l'implémentation, existence d'une norme internationale concernant la définition du langage. Conformité de l'implémentation par rapport à la norme, existence de plusieurs fabricants pour le même langage.
\item[$\bullet$]La documentation du langage
\end{itemize}
Connaître un langage de programmation est un atout de plus en plus important sur le marché du travail, puisque la demande de développeurs de logiciels augmente de façon exponentielle ces dernières années.
Cependant, quand on débute dans la programmation, on peut être confus face aux centaines de langages que l’on peut choisir. C’est pour cela que je rédige cet article afin de conseiller les futurs programmeurs dans le choix du langage idéal pour eux.
  % Contenu de l'introduction

\section{Qu’est-ce qu’un langage de programmation ?} % section 2
Un langage de programmation est un langage permettant de formuler des algorithmes et de produire des programmes informatiques qui appliquent ces algorithmes.

Un programme est un enchaînement d'instruction, écrit dans un langage de programmation, exécutée par un ordinateur, permettant de traiter un problème et de renvoyer des résultats. Il représente la traduction un algorithme à l'aide d'un langage de programmation.
\begin{figure}[h] % insère une figure ici (h = "here")
  \centering % centre la figure
  \includegraphics[scale=1]{img1.PNG} % insère une image en taille réelle
\end{figure}
L'ordinateur est une machine totalement dénuée d'intelligence. Un programme exécute les instructions bien précises c'est-à-dire celle que le programmeur lui a donnée. Des erreurs ou des fantaisies lors de son exécution ne proviennent pas de l'ordinateur mais d'une erreur de conception.

\section{Historique des langages de programmation} % section 3

Un ordinateur ne connaît que le système de numération binaire. Un langage utilisant le système binaire s'appelle langage machine.
Pour écrire des programmes sous des formes accessibles,les langages d'assemblage ont vu le jour au début des années 50, 
 
Cependant, un programme écrit en langage d'assemblage n'est pas directement exécutable par la machine. Il doit être traduit en un programme équivalent en langage machine. Cette opération de traduction s'effectue grâce à un autre programme appelé assembleur.
Le langage d'assemblage présente un inconvénient : il reste lié à l'ordinateur pour lequel il a été écrit car chaque famille de processeurs possède son propre langage d'assemblage. Il est difficile à utiliser car il nécessite de bonnes connaissances sur le fonctionnement des processeurs.
C'est pourquoi furent conçus les langages de programmation dits évolués plus compréhensibles et plus lisibles par l'homme.
Un langage de programmation est défini par des règles d'écriture des règles de construction que doivent respecter les programmes. La difficulté, pour le programmeur, consiste à respecter ses règles imposées en fonction des differents paradigmes de programmation.

On distingue 4 grandes générations de langages de programmation:
\begin{itemize}
   \item[$\bullet$]Langages machine.
   \item[$\bullet$]Langages symboliques et autocodes.
   \item[$\bullet$]Langages indépendants du matériel, comme Basic, C, Cobol, Algol...
   \item[$\bullet$]Langages conçus pour décrire le problème, comme Simula et autres langages à objets .
\end{itemize}
   Nouvelles tendances 
\begin{itemize}   
\item[$\bullet$] La cinquième génération pourrait être celle des langages Internet, donc fonctionnant sur toute
machine et compilés en code intermédiaire (dit virtuel).
\item[$\bullet$] Les langages "Markup" inspirés de XML sont la dernière tendance, ils intègrent le code et les
données sous une forme extensible, et qui fonctionnent sur le web.
\end{itemize}
Assembleur
 
 Les assembleurs existent depuis le début des ordinateurs. Ils étaient encore utilisés dans les années 70. Ce langage peu gourmand en ressources, convenait parfaitement aux ordinateurs de l’époque mais n’est plus utilisé aujourd’hui.
 
 Indépendamment de ces générations, les grandes dates sont les suivantes:
 \begin{itemize}
\item[$\bullet$] Années 50: Création des langages de haut niveau dont Autocode, IPL, Fortran
\item[$\bullet$] Années 60: Foisonnement de langages spécialisés. Forth. Simula I. Lisp, Cobol.
Certains langages généraux: Algol, PL/1, ne s’imposent pas.
\item[$\bullet$] Années 70: Duel entre programmation structurée avec Pascal et l'efficacité du langage C
 (cela dure encore en 2000). Généralisation du Basic interprété sur les micro-ordinateurs
apparus en 1977, jusqu'à la fin des années 80.
\item[$\bullet$] Années 80: Expérimentation d'autres voies et notamment des objets. ML. Smalltalk. Sur les
micro-ordinateurs, on utilise maintenant C, Pascal, Basic compilé.
\item[$\bullet$] Années 90: Généralisation de la programmation objet grâce aux performances des microordinateurs. Java, Perl, Python s'ajoutent aux langages micros.
\item[$\bullet$] Années 2000: Programmation Internet (et les innovations à venir).
\item[$\bullet$] Années 2010: Concurrence et asynchronisme. Les langages JavaScript, Go, Julia entre autres
aident à créer des applications en ligne fluides.
\end{itemize}


Nombre de langages dont + de 100 (recensés) ont vu le jour depuis les années 50. La plupart n’ont fait qu’une apparition. D’autres comme le Cobol ou le Fortran restent d’actualité, car bien qu’ils ne soient plus utilisés aujourd’hui, de grosses applications ont été développées par des entreprises et sont toujours en fonctionnement. Leur re-développement sous de nouveaux langages représente des investissements non négligeables que certaines entreprises hésitent à mettre en œuvre.
\end{document} % fin du corps du texte